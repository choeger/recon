% sample file for Modelica Conference paper

\documentclass[11pt,a4paper,twocolumn]{article}
\usepackage{graphicx}
\graphicspath{{fig/}}
\usepackage[T1]{fontenc}
\usepackage[utf8]{inputenc}    %% european characters can be used
\usepackage{lmodern,amsmath,mathptmx,url}      %% recommended for readable pdf
\pagestyle{empty}                %% no page numbers!
\usepackage{geometry}            %% please don't change geometry settings!
\geometry{left=20mm, right=20mm, top=25.4mm, bottom=25mm, noheadfoot, columnsep=8mm}
%\parindent0pt
\parindent10pt

\usepackage[backend=bibtex,sorting=none]{biblatex}
\addbibresource{recon.bib}

% some additional packages
\usepackage{listings} % for code listings
\usepackage{color}
\usepackage[hidelinks=true]{hyperref}
\hypersetup{%
  pdftitle = {recon  -- Web and network friendly simulation data formats},
  pdfauthor = {Michael Tiller and Peter Harman},
  pdfsubject = {10th International Modelica Conference 2014},
  pdfkeywords = {Modelica, FMI, simulation results, cloud, web, open source}}


\lstdefinelanguage{JavaScript}{
  keywords={typeof, new, true, false, catch, function, return, null, catch, switch, var, if, in, while, do, else, case, break},
  keywordstyle=\color{blue}\bfseries,
  ndkeywords={class, export, boolean, throw, implements, import, this},
  ndkeywordstyle=\color{darkgray}\bfseries,
  identifierstyle=\color{black},
  sensitive=false,
  comment=[l]{//},
  morecomment=[s]{/*}{*/},
  commentstyle=\color{blue}\ttfamily,
  stringstyle=\color{red}\ttfamily,
  morestring=[b]',
  morestring=[b]"
}

% usefull commands
\newcommand{\myr}{\textsuperscript{\textregistered}}
\newcommand{\ud}{\mathrm{d}}
\newcommand{\matx}[1]{\mathbf{#1}}
\newcommand{\recon}{\texttt{recon}}
\newcommand{\wall}{\texttt{wall}}
\newcommand{\meld}{\texttt{meld}}
\newcommand{\msgpack}{\texttt{msgpack}}
\newcommand{\code}[1]{\texttt{#1}} % make quoting code text a bit simpler

\begin{document}

\title{\recon\  -- Web and network friendly simulation data formats}

\author{Michael Tiller\\
  \href{http://xogeny.com}{Xogeny Inc.}, USA\\
  \href{mailto:michael.tiller@xogeny.com}
       {\nolinkurl{michael.tiller@xogeny.com}}
  \and Peter Harman\\
  \href{http://www.cydesign.com}
       {CyDesign}, UK\\
  \href{mailto:peter@cydesign.com}{\nolinkurl{peter@cydesign.com}}}
\date{} % <--- leave date empty
\maketitle\thispagestyle{empty} %% <-- you need this for the first page

\section*{Abstract}

There are many different commonly used file formats for storing time
series data.  Most of these file formats are designed with the
assumption that the file itself will be locally available to the
software that will be reading or writing the data stored in them.
While this assumption is an excellent one for desktop based tools with
direct access to disk drives capable of moving virtually
instantaneously around from sector to sector, there are a growing
number of applications for which local access is not necessarily
available.  For these applications, we've initiated the
\recon\ project to develop more suitable formats.

With the emergence of web and cloud based modeling and simulation
technologies, the time has come to explore file formats specifically
optimized for non-desktop applications.  In this paper, we present a
new set of file formats that are specifically designed for web and
cloud based approaches.  This paper reviews the key requirements for
web and cloud enabled applications and then presents a specification
for two file formats that address those requirements.

When considering the various use cases that drove our requirements, we
recognized that two different file formats were actually required.
The first format, the \wall\ format, is optimized for writing.  The
other format, the \meld\ format, is optimized for reading over a
network (\textit{i.e.,} minimizing the number of reads and bytes
read).  We will discuss the layout of each of these formats and
describe the use cases for which they are most appropriate.

In the open tradition of the Modelica Association, the authors have
made specifications and implementations for these formats available as
open source libraries with the hope that they will benefit the
community as a whole.

\paragraph{Keywords:}\emph{Modelica, FMI, simulation results,
  cloud, web, open source}

\section{Introduction}
\label{sec:intro}

Several groups have examined the issue of standardized file
formats\cite{PfeifferHDF5,DyMatHDF5} in the context of Modelica.  In
keeping with the principles of the Modelica Association, an ideal
choice would be a production ready format that is open source and
cross-platform.  With these requirements in mind, most people consider
HDF5 a natural choice.  There are already open source implementations
and the file format has been widely used.  In fact, it has even been
adopted by The MathWorks for use in MATLAB.

But HDF5 has some practical drawbacks.  The first is that it is not
truly cross-platform.  The reference implementation of HDF5 is written
in C.  The implementation is primarily targeted for use within C, C++
or Fortran applications.  While there are various libraries available
for reading HDF5 on the Java platform\cite{HDFJava}, they are
incomplete and awkward to use.

Another issue with HDF5 is that for simple time-series data it is over
engineered.  HDF5 is feature rich, of that there is no question, but
these features come at the cost of complexity.  This is why you see
very few implementations outside of the reference implementation.
Furthermore, the file format makes extensive use of ``seek''
operations and assumes they are relatively inexpensive.  This
assumption is reasonable if you are able to communicate directly with
the hard drive that the files are stored on, but it isn't reasonable
when these files are only available through the network.

There are, of course, many standards for encoding data in web or cloud
based environments.  The most popular formats, by far, are Javascript
Object Notation (or JSON, for short) and XML.  There are other
approaches as well like Google Protocol Buffers\cite{GPB},
Avro\cite{Avro} and Thrift\cite{Thrift}.  Different approaches have
different goals.  Some define schemas to add a level static checking.
Others exist mainly to compress the data being transmitted.  Finally,
others exist to introduce a layer of interoperability with RPC
frameworks or ``big data'' tools like Hadoop\cite{Hadoop} and
Storm\cite{Storm}.

So where does that leave us?  Should we adopt the tried and true
standards from the engineering world and simply live with their lack
of interoperability with important platforms like Java or Javascript
and poor performance when accessed remotely?  Or, should we adapt
tools from the ``big data'' world, that were developed for quite
different use cases, to work in the engineering world.

In some sense we've chosen a compromise.  As we will see shortly, the
\wall\ and \meld\ formats are fundamentally derived from the
\msgpack\cite{msgpack} specification.  This gives us excellent cross
platform compatibility.  But \msgpack\ is simply a serialization
protocol.  To address some of our more important concerns, to be
discussed shortly, we needed to design a format the imposed additional
structure on top of \msgpack. So in this sense, we've created a set of
original file formats that leverage open standards, like \msgpack, but
re-purpose them for modeling and simulation applications.

\section{Goals}
\label{sec:goals}

After independently reviewing various file formats, the authors were
not happy with the existing options for web and cloud based modeling
and simulation.  The \recon\ project started as a discussion about
requirements.  For our applications, the following requirements were
identified:

\subsection{Requirement 1 - Adding Data}

% 1) It should be easy to append data to a file.

Simulations are constantly producing additional data.  For this
reason, adding new data to an existing file is an operation performed
many times during a simulation.  For this reason, adding data to an
existing file should be fast and easy.  The key thing is to avoid having to
rewrite previous data or, even worse, move data around within the
file.  For this reason, the ideal solution is to have the ability to
simply append new data at the end of the file.

\subsection{Requirement 2 - Minimizing I/O}

% 2) It must be possible to extract individual signals with a minimum of I/O.

In web and cloud based applications, it is not always practical to
download the complete set of results for a simulation into the browser
environment.  There are many use cases where it would be best to be
able to access results ``on demand''.  In these environments, such
requests for data will be done via HTTP\cite{HTTP}.  However, each of
these requests will come with far greater latency than a simple
request to read from a disk drive and far less bandwidth.  As such, we
would want to minimize the number of such requests and the amount of
data necessary to transmit in each request.  This means we need a way
to ``cluster'' the data we are interested in so as to minimize both
the number of requests required and the amount of data in each
response.

\subsection{Requirement 3 - Cross Platform Support}

% 3) The format must work across languages.

The file formats developed as part of the \recon\ project are targeted
at web and cloud based applications.  Client side web programming is
dominated by Javascript.  On the other hand, server side programming
is done in a wide variety of languages (\textit{e.g.,} Java, Python,
Javascript).  Meanwhile, numerical analyses such as simulation are
typically done in languages like C, C++ and FORTRAN.  For this reason,
it should be possible to implement libraries in all of these languages
for generating and extracting simulation results.

\subsection{Requirement 4 - Aliasing}

% 4) It must be possible to define aliases.

One of the common patterns in component-oriented modeling approaches
like Modelica is that many variables end up with exactly the same
solution trajectories.  When storing simulation results from such
tools, it is useful to recognize that considerable disk space can be
saved by recognizing the fact that these variables all share a common
underlying solution trajectory.  Typically, the data for each unique
solution trajectory is stored once and each variable is simply a
reference to the underlying solution trajectory.  Even more storage
can be saved by recognizing that some trajectories are related to
other trajectories by very simple transformations (\textit{e.g.,} a
simple sign change).  For this reason, it is very useful if these
kinds of relationships are directly represented in the file format
itself.

\subsection{Requirement 5 - Data Types}

% 5) It should be possible to store "structures/dictionaries" as well
% as regular array types.

When dealing with simulation results that come from the solution of
differential equations, the main type of result is a solution
trajectory.  In these cases, both the dependent and independent
variables are typically represented as floating point numbers.

But these are not the only types of data a simulation or other
numerical analysis might yield.  From the Modelica world, we might
easily have results that are either reals, integers, booleans or strings
(since these are all fundamental built-in types in Modelica).  But why
not hierarchical data structures (as represented by records in
Modelica) as well?

\subsection{Requirement 6 - Metadata}

% 6) It must be possible to associate metadata with entities in the
% file.

One issue with data files is that if you don't provide a means for
associating metadata with entities in the file, the metadata will
\textbf{become} entities in the file.  For this reason, we deemed it
important that metadata should be treated in a ``first-class'' way.
Specifically, it should be possible to associate metadata with the
file as a whole and with data structures in the file all the way down
to individual signals.  This would allow tools to persist other
important information, beyond the solution, in these data files.  For
example, information about common plots or plotting options,
descriptions of the signals, units or display units could all be
managed in a structured way without being confused with data and
without needing to be a formal part of the file format specification.

\subsection{Requirement 7 - Hierarchy}

% 7) It should be possible to organize information hierarchically.

Many tools create structures that are hierarchical.  In the Modelica
world, we have deep hierarchies of instances in simulated models.  We
also have hierarchies for packages and the definitions contained in
them.  So it is important that a file format can represent these
hierarchies in some way.

In our experience, trying to organize results according to an instance
hierarchy creates quite a bit of complexity.  While tools could
exploit some of the previous requirements (primarily 5 and 6) to
achieve a hierarchical representation, we've found that simply
encoding hierarchy in the names of variables (\textit{e.g.,}
\code{car.engine.crankshaft.tau}) is typically sufficient and can
avoid considerable complexity.

\subsection{Requirement 8 - Easy Translation}

% Support for heterogenous data (no strict type constraints)

Even though our goal is to have a format that is well suited to web
and cloud based applications, it should also be capable of
representing the kinds of simulation results we interact with in a
desktop environment.  For this reason, one of our requirements was the
ability to translate data in the ``dsres'' format into the
\recon\ formats.  Such a translation should preserve all data and
metadata normally associated with the ``dsres'' format.  Furthermore,
the resulting \recon\ files should be approximately the same size as
the original ``dsres'' file.

\section{Approach}

\subsection{Reading vs. Writing}

In reviewing these requirements, the main design challenge was trying
to reconcile requirements 1 and 2.  Implementing requirement 1
typically involves the need to write data out one row at a time, where
each row represents the values of all the variables for successive
solution times.  As such, the solution values for any particular
\textit{variable} are widely spaced.  However, requirement 2 requires
us to be able to extract a given variable with a minimum number of I/O
operations.  In other words, requirement 1 typically results in data
being fragmented while requirement 2 depends on that data being
clustered together.

% Two formats
% clocks

Our solution to this design problem was to design two file formats.
The first, the \wall\ format, is designed for writing.  Not only does
it make adding data fast and efficient, it also supports, unlike the
\code{dsres} approach, adding data for multiple tables at once.  In
practice, this means that if you have variables in your simulation
that are partitioned such that they have different independent
variables (as with the new clock semantics in Modelica 3.3), this
format supports writing out new data for any of these variables.  In
other words, you can add results that may have completely different
time bases.  You can also include results from multiple simulations
and use the metadata features to associate the specific parameter sets
with each table.

The other format, the \meld\ format, is optimized for reading.
Specifically, it is optimized for requirement 2.  For an ideal
format, it should be possible to extract a single result trajectory in
a single read.  This would act to minimize the impact of latency.
Furthermore, the bytes read should contain only data associated with
the desired signal.  This would minimize the impact of limited
bandwidth.  As we will discuss shortly, the \meld\ file manages to
achieve these performance characteristics for all but the first signal
read.

Our expectation is that tools will write data out (during simulation)
in the \wall\ format.  Tools may choose to keep the data in this
format.  For platforms where network access is not a requirement, the
write optimized nature of the \wall\ format will probably be adequate
for both reading and writing.  However for cases where data will be
read over a network, we expect that tools will, upon completing a
simulation, rewrite their data into the \meld\ format.

\subsection{Serialization}

% Started with BSON, switched to MsgPack

There are really two aspects to each \recon\ format.  The first is the
structure of the file (where different pieces of information reside in
the file, something we'll discuss in Section \ref{sec:spec}) and the
other is how the actual data is represented.

Obviously, the data is represented as individual bytes.  So we must
define the process by which multi-byte pieces of information
(\textit{e.g.,} floating point numbers) are ``serialized'' into bytes.
One of the implicit goals of this project was to make a file format
that was easy to read and write.  Since serializing and deserializing
data was a big part of the implementation, we could make the
implementation much easier if we leveraged existing standards for
serialization and deserialization.

One of the interesting things about coming from the web and cloud
based application side is the ubiquity of JSON notation.  While the
Javascript language itself has many ``unusual'' semantics, the syntax
and semantics around serialization and deserialization are
surprisingly simple and intuitive.  Unlike XML, for example, writing a
parser for JSON and then mapping into a native language representation
is surprisingly easy and widely supported.

However, JSON is a textual representation.  The problem with a textual
representation is the additional overhead of having to parse and
interpret the text and convert, without any loss of precision, into a
binary representation.  For this reason, we didn't consider JSON by
itself a practical approach to serialization and deserialization.

While we wanted to avoid the parsing aspect of JSON, the JSON data
model\cite{JSON} is well suited to our purposes and many different
groups have attempted to create a binary representation that follows
the JSON data model.  So our initial approach was to consider BSON
\cite{BSON} which is a binary format that is formally specified and
widely implemented because it is one of the cornerstones of the
MongoDB database\cite{MongoDB}.

Unfortunately, the BSON serialization scheme has a significant
drawback.  The way it serializes arrays is very space inefficient.
This is because JSON itself supports sparse indexing of arrays.  As a
result, a serialization must include, for each value in the array, the
index as well.  This adds significant overhead.  There is no way to
specify that all elements in the array are sequential.  As such, there
is no way to avoid this significant penalty.

Fortunately, our reference implementation in Python\cite{pyRecon} had
a clean separation between the serialization scheme and the structural
aspects.  This made it very easy for us to experiment with other
serialization techniques.  We investigated other similar serialization
schemes like Smile\cite{Smile}, BJSON\cite{BJSON} and
UBJSON\cite{UBJSON}.  However, none of them seemed to have a critical
mass behind them.  Indeed, there doesn't seem to be a consensus in the
JSON community on how to serialize JSON in a binary form.

As part of our investigation, we also looked at \msgpack.  It turned
out that, like BSON, \msgpack\ was formally specified and implemented
for a wide variety of platforms\cite{msgpack}.  Furthermore, in
testing \msgpack, we found that it had much better storage efficiency
compared to BSON.  So, in the end, we moved forward using the
\msgpack\ serialization scheme.

The \msgpack\ approach had a couple of unanticipated benefits.  In
\msgpack, floating point numbers can be encoded in either single or
double precision representations.  Also, the specification identifies
and formally specifies several different optimizations to minimize the
number of bytes required to store short integers or short strings.
The underlying ``types'' permitted in this format map very easily into
the JSON format which, in turn, means that it maps well into the
native data types common across all the languages we are interested
in.  Finally, the \msgpack\ serialization scheme includes an extension
mechanism for including additional data types beyond those in the
specification.  While we don't have any immediate use for these
extensions, it is nice to have that feature if we ever find that
\msgpack's serialization is too constraining.

\section{Specification}
\label{sec:spec}

With the motivation behind us, let's turn to the actual specification
of these formats.  In this section we will describe the layout of both
the \wall\ format and the \meld\ format.  As mentioned in the previous
section, the serialization is done using \msgpack.  So we will focus
mainly on the layout of data within the file.  When describing the
actual data being stored we will use JSON notation to document the
data with the implicit understanding that this data will be serialized
and deserialized using \msgpack.

% tables and objects

Before we get into the specifications for each of these formats, it
will be useful to discuss a few topics that are relevant to both
formats.  For example, what exactly are we storing in these file
formats?  Both formats support the storing of tables and objects.
Tables are, as the name implies, a structure for storing tabular data.
It is worth noting that there are no restrictions on what kind of data
can be stored in a table except those imposed by the underlying
\msgpack\ format (which are very few).  This means tables can mix
integers, floats, doubles, longs, short ints, strings, booleans and
even objects across different columns in the same table.

In addition to tables, we can register objects to be stored in our
results file.  These are essentially free-format pieces of data.  The
objects can be used to create arbitrarily deeply nested data
structures that mix all varieties of data.  Once again, the only
limitations are those imposed by \msgpack\ and/or the source language.

When storing tables and objects in a file, they must be referred to by
names.  The same namespace is used for both objects and tables (in
other words, you cannot have a table with the same name as an
object).  It also means that no two tables and no two objects can
share the same name either.  The columns of each table are also named
and no two columns within the same table can share the same name.  But
there is no such restriction between columns in different tables (or
fields in different objects, for that matter).

Finally, it is worth noting that certain pieces of data are optional.
In those cases, we have consistently followed a policy of leaving out
both the key and the value.  In other words, it is not sufficient to
simply associate a \code{null} value with a key.  \textbf{Both the key
and the value must be removed} if there is no value provided.  The
guiding principle here is that parsers should not have to do excessive
amounts of null value checking.

With that background out of the way, let's proceed with our
explanation of the two file formats.

\subsection{Wall Format}
\label{sec:wall_spec}

Recall that the \wall\ format is optimized for writing and that this,
in turn, means being able to easily add data.  You can think of the
\wall\ format as being similar to a brick wall where each brick (new
piece of data) is staggered with respect to others.  As you will see,
we are \textbf{not} storing information in homogenous arrays and this
means that we cannot predict the index of data simply based on
information about which row or column it is in.  Also note that it is
possible to have data from two different tables interleaved between
each other.  This allows us to add data with two distinct time bases.
But it makes the location of data even more difficult to predict.
However, remember that the \wall\ format is optimized for writing, not
reading and that if network access is required then tools will
typically rewrite their data into the \meld\ format.

\subsubsection{Leading Bytes}

Each \wall\ file starts with the following sequence of bytes:

{\footnotesize
\begin{verbatim}
0x72 0x65 0x63 0x6f 0x6e 0x3a 0x77
0x61 0x6c 0x6c 0x3a 0x76 0x30 0x31
\end{verbatim}
}

This is a hex encoding of the ASCII string \code{recon:wall:v01}.
This allows us to identify whether this is a \recon\ \wall\ file and, if
so, what version of the specification should be applied.

The next four bytes are a binary encoding of the length of the header.
This encoding is done in so-called ``network byte order''
(big-endian).  The byte encoding of the length is not considered part
of the header (\textit{i.e.,} the length indicated doesn't include the
4 bytes that encode the length).

\subsubsection{Header}

Once the length of the header is known, the bytes for the header are
read in.  These bytes are assumed to have been serialized in
\msgpack\ format so we must next unpack (deserialize) these bytes.
Once unpacked, the header should contain the following information:

{\footnotesize
\begin{verbatim}
{
  "fmeta": {<file-level metadata>},
  "tabs": {
    "<table name>": {<table data>}
  },
  "objs": {
    "<object name>": {<object metadata>}
  }
}
\end{verbatim}
}

The \code{"fmeta"} key is associated with the value for any file level
metadata.  This metadata is, itself, represented in our notation here
as an object in JSON but will be encoded as a map in \msgpack.  The
\code{"tabs"} key is associated with a value that maps the names of
tables to table data.  The format of this table data is as follows:

{\footnotesize
\begin{verbatim}
{
  "tmeta": {<table-level metadata>},
  "sigs": [<list of signals>],
  "als": {
    "<aliasname>": {
      "s": <base signal name>,
      "t": <transform string> // OPTIONAL
    }
  },
  "vmeta": {
    "<varname>": {
      <variable-level metadata
    } // OPTIONAL
  }
}
\end{verbatim}
}

The \code{"tmeta"} key is associated with metadata (again, represented
as a \msgpack\ map) but this time it is metadata associated with the
table.  In addition, we have the \code{"sigs"} key which represents an
ordered list of signals.  A signal represents an actual solution
trajectory and the order is important because the order in this list
indicates the order in which the data will be stored in successive
rows (to be discussed shortly).

The \code{"als"} key represents any aliases present in the table.
Again, this is a \msgpack\ map where the name of the alias is the key.
There are two essential pieces of information associated with each of
the aliases.  The first, stored under the \code{"s"} key (which stands
for ``signal'') is the name of the base signal that this alias is
based on.  The \code{"t"} key is used to represent the transformation
that should be applied to the base signal to compute the value of the
alias signal.  Note that this transformation \textbf{is optional}.
The possible values will be described shortly in \ref{sec:trans}.

Finally we have the \code{"vmeta"} key, which is a map where the keys
are the names of variables (\textit{i.e.,} both signals and aliases)
and the values are any metadata (again, stored as a \msgpack\ map)
associated with the named variable.  Note that keys are only present
in the \code{"vmeta"} map if there is metadata associated with that
variable.

Returning to the header level entries, the \code{"objs"} key is
associated with a value which is, in turn, a \msgpack\ map.  Each key
in that map represents an object name and the value associated with
those object name keys represents the metadata associated with the
named object.

\subsubsection{Entries}

Following the header, the remainder of the file consists of
``entries''.  Each entry is preceded by 4 bytes in network byte order
indicating the length of the entry (again, the length indicated does
not include the 4 bytes used to represent the length).  All entries
are encoded maps in \msgpack\ format.  There are two types of possible
entries and there are no rules about which can be present
(\textit{i.e.,} they can appear in any order and be interleaved).

The first type is a ``row entry'' which details a new row for a
specified table.  The format of a row entry is as follows:

{\footnotesize
\begin{verbatim}
{
  "<table name>": [list of signal values]
}
\end{verbatim}
}

where the table name must be a key in the \code{"tabs"} map found in
the header and the order of values in the list of signal values must
correspond to the order defined by the list associated with the
\code{"sigs"} key value within that table.

The other entry type is a ``field entry''.  These represent updates to
the values of fields in objects and have the following format:

{\footnotesize
\begin{verbatim}
{
  "<object name>": {
    "<field name>": <field value>,
    "<field name>": <field value>
  }
}
\end{verbatim}
}

Note that the header does not specify which fields are present in the
object.  This can only be determined by processing all the field
entries and incorporating fields as they are given values.  The
complete value for a given object is determined by processing all
field entries associated with that object in the order they appear in
the \wall\ file.  The complete value is then simply the final value
after all processing is completed.  One consequence of this is that it
is therefore possible to have the values of individual fields change
while writing the file.

\subsubsection{Wall Format Summary}

% Summary

The following outline attempts to summarize all the details presented so
far:

{\footnotesize
\begin{verbatim}
// ID
0x72 0x65 0x63 0x6f 0x6e 0x3a 0x77
0x61 0x6c 0x6c 0x3a 0x76 0x30 0x31
// Header length, network order
0x?? 0x?? 0x?? 0x??
{
  "fmeta": {<file-level metadata>},
  "tabs": {
    <table name>: {
      "tmeta": {<table-level metadata>},
      "sigs": [<list of signals>],
      "als": {
        <aliasname>: {
          "s": <base signal name>,
          "t": <transform string> // OPTIONAL
        }
      },
      "vmeta": {
        <varname>: {
          <variable-level metadata>
        } // OPTIONAL
      }
    },
  },
  "objs": {
    <objname>: {<object metadata>}
  }
}

// Followed by zero or more entries
// which can be either...

// ...field entries...
0x?? 0x?? 0x?? 0x?? // entry length
{
  <object name>: {
    <field name>: <field value>,
    <field name>: <field value>
  },
}

// ...or row entries
0x?? 0x?? 0x?? 0x?? // entry length
{
  <table name>: [list of signal values]
}
\end{verbatim}
}

where all maps are encoded in \msgpack.

\subsection{Meld Format}
\label{sec:meld_spec}

\subsubsection{Leading Bytes}

Each \meld\ file starts with the following sequence of bytes:

{\footnotesize
\begin{verbatim}
0x72 0x65 0x63 0x6f 0x6e 0x3a 0x6d
0x65 0x6c 0x64 0x3a 0x76 0x30 0x31
\end{verbatim}
}

This is a hex encoding of the ASCII string \code{recon:meld:v01}.
This allows us to identify whether this is a \recon\ \meld\ file and, if
so, what version of the specification should be applied.

The next four bytes are a binary encoding of the length of the header.
This encoding is done in so-called ``network byte order''
(big-endian).

\subsubsection{Header}
\label{sec:meld_head}

Once the length of the header is known, the bytes for the header are
read in.  These bytes are assumed to have been serialized in
\msgpack\ format so we must next unpack these bytes.  Once unpacked,
the header should contain the following information:

{\footnotesize
\begin{verbatim}
{
  "fmeta": {<file-level metadata>},
  "tabs": {
    "<table name>": <table data>
  },
  "objs": {
    "<object name>": <object data>
  },
  "comp": true|false // Compression flag
}
\end{verbatim}
}

This is very similar to the \wall\ format presented in Section
\ref{sec:wall_spec}.  Again, we see file level metadata exactly as it
is used in the \wall\ format.  We also have the \code{"tabs"} and
\code{"objs"} keys, also present in the \wall\ format but with an
important distinction which is that the values that follows them have
a different format, as we shall see shortly.

But we also have a new key, the \code{"comp"} key, which isn't present
at all in the \wall\ format.  The value associated with the
\code{"comp"} key indicates whether the remainder of the file (after
the header) is not just encoded (using \msgpack) but also compressed.  If
the value associated with the \code{"comp"} key is \code{true}, then
all remaining \msgpack\ encodings present in the file after the header
are compressed using bz2\cite{BZ2} compression.

For reasons that will become obvious, the data for tables and objects
is different in the \meld\ header than in the \wall\ header.  In a
\meld\ file, the table data has the following format:

{\footnotesize
\begin{verbatim}
{
  "tmeta": {<table level metadata>},
  "vars": <list of variable names>,
  "toff": {
    "<varname>": {
      "i": <index of variable data>,
      "l": <length of variable data>,
      "t": <transform string> // OPTIONAL
    }
  },
  "vmeta": {
    "<varname">: {
      <variable level metadata
    } // OPTIONAL
  }
}
\end{verbatim}
}

The \code{"tmeta"} is, again, the table level metadata.  Similarly,
the \code{"vmeta"} key is associated with variable level metadata
which is a map where the variable name is the key (again, only present
if there is metadata associated with the specified variable) and the
associated value is the variable level metadata.

The \code{"vars"} key is associated with an ordered list of the
variables present in the file.  The \code{"toff"} key is associated
with a map that specifies important information about the location of
the variables within the file.  It is the \code{"toff"} data that
makes it easy for us to extract individual signals.  The \code{"i"}
key is associated with the starting byte, within the file (starting
from 0), of the data associated with the variable and the \code{"l"}
key is associated with the length of that data.  The optional
\code{"t"} key defines the transformation, if any, to be applied to
the variable data (see Section \ref{sec:trans} for more details).

Returning to the header data, the object data associated with the
\code{"objs"} key has the following format in a \meld\ file:

{\footnotesize
\begin{verbatim}
{
  "ometa": {<object level metadata>},
  "i": <index of object data>,
  "l": <length of object data>
}
\end{verbatim}
}

The \code{"ometa"} key is associated with the metadata of the
associated object.  The \code{"i"} and \code{"l"} keys are used just
as they are within tables, to define the index and length,
respectively, of the object data within the file.

\subsubsection{Variable Data}

As discussed in Section \ref{sec:meld_head}, both variables
(conceptually, columns in tables) and objects have an offset and a
length provided in the header.  In the case of a variable, the data
that is extracted from that location in the file will be a
\textbf{list} of values in \msgpack\ format.  The values in that list
represent the values for the specified solution variables (first row
first, last row last).  In the case of an object, the data that is
extracted from that location in the file will be a \textbf{map} where
the keys in the map represent the fields present in the object and the
values associated with those keys are the field values.

\subsubsection{Header Size}

% Shrinkage

It is worth pointing out that when writing the file, the exact length
of the header (when encoded in \msgpack\ format) cannot be known
\textit{a priori} (we shall explain why, shortly).  For this reason, a
collection of bytes representing the largest possible size must be
reserved for the header.  The size of the header depends on the number
of tables and objects as well as the number of signals in each table
so it important that all of this information is known before
determining the maximum number of bytes required to represent the
header.

However, when the final version of the header is written out (once the
location of all the data can be determined), its size may be less than
originally anticipated.  This is a result of the \msgpack\ format's
aggressive compression of small integers.  For this reason, there may
be a few unused bytes present between the end of the header and the
first variable or object data in the file.  While it is true that a
few bytes will be wasted as a result, it really only means that
\msgpack's aggresive optimizations will be wasted in this case (and
this case only).

\subsubsection{Meld Format Summary}

The following outline attempts to summarize all the details presented so
far:
{\footnotesize
\begin{verbatim}
// ID
0x72 0x65 0x63 0x6f 0x6e 0x3a 0x6d
0x65 0x6c 0x64 0x3a 0x76 0x30 0x31
// Header length, network order
0x?? 0x?? 0x?? 0x??
{
  "fmeta": {<file-level metadata>},
  "tabs": {
    "<table name>": {
      "tmeta": {<table level metadata>},
      "vars": <list of variable names>,
      "toff": {
        "<varname>": {
          "i": <index of variable data>,
          "l": <length of variable data>,
          "t": <transform string> // OPTIONAL
        }
      },
      "vmeta": {
        "<varname">: {
          <variable level metadata
        } // OPTIONAL
      }
    }
  },
  "objs": {
    "<object name>": {
      "ometa": {<object level metadata>},
      "i": <index of object data>,
      "l": <length of object data>
    }
  },
  "comp": true|false // Compression flag
}

// Followed by any padding
// resulting from header shrinkage

// Followed by zero or more blocks
// of msgpacked data representing
// either vectors or objects whose
// offsets and lengths are specified
// in the header)
\end{verbatim}
}

\subsection{Transformations}
\label{sec:trans}

% Discuss transformation grammar

In the previous sections, there were several mentions of a so-called
``transform string''.  This is an optional piece of information
associated with an alias (in the case of the \wall\ format) or a
variable (in the case of the \meld\ format).  It defines the
transformation, if any, that must be performed over some base data in
order to retrieve the true value of the referenced data.  There are
presently only two allowed transformation types that are supported by
the \recon\ formats.

The first transformation type is the ``inverse'' transformation.  This
transformation is indicated when the transform string has a value of
\code{"inv"}.  The impact of the inverse transformation depends on the
data type.  For numeric data types, the inverse transformation causes
the data to have its sign inverted.  For boolean data, the inverse
transformation applies a logical not operation to the data.  With this
simple transform alone, it is possible to avoid storing a significant
amount of data.

The other transform type is the ``affine'' transform.  This
transformation is indicated when the transform string has a value of
\code{"aff(s,o)"}, where \code{s} represents a scale factor and
\code{o} represents an offset value.  In the presence of this
transformation, all \textbf{numeric} values in the base data should be
multiplied by the scale factor, \code{s}, and then added to the offset
value, \code{o}.  As one reviewer of this paper noted, unit transforms
are almost always affine in nature.  This means that the affine
transformation permits results to be stored in many different physical
units without taking up any appreciable additional storage.

No transform should be applied to the data if:
\begin{itemize}
\item No transform string is present
\item The transform string is unrecognized/cannot be parsed
\item The transform does not apply to the underlying data type
  (\textit{e.g.,} applying the \code{affine} transformation to a
  Boolean value)
\item There was an kind of error or exception when attempting to apply
  the transformation
\end{itemize}
In all but the first case, the tool or environment is strongly
encourage to provide an error message alerting the user.  Tools are
also free to treat all but the first of these conditions as an error
and suppress access to the alias data.

\section{Discussion}
\label{sec:discussion}

\subsection{Use Case}

Although it is implicit in the requirements listed in Section
\ref{sec:goals}, it is worth elaborating a bit more on the specific
use case that drove these requirements.

For web and cloud based simulation, the bulk of the computational work
is done remotely.  In cloud services, there is an effect sometimes
called ``data gravity'' which dictates that to improve overall
throughput, the computing platforms tends to gravitate to where the
data is stored (\textit{i.e.,} the same service provider or at least
ones that are connected with low latency, high bandwidth connections).

So if one seeks an optimal configuration where the computing and
storage are well connected, it is easy to store the complete
simulation results without any significant concerns for latency or
bandwidth.

The complication comes from having to access that data via a
``normal'' network connection.  A typical use case (and the one that
we imagined while formulating our requirements) is one where a web
application, running in a web browser, needs to display simulation
results.  The question then is how could such an application make
effective use of simulations results generated ``in the cloud''?

One approach to this could be for the web browser to download the
complete simulation results.  Without consideration for web and cloud
based applications, such results might very likely be stored in
formats like HDF5 or MATLAB version 4 format.  The first problem is
that these formats are not well supported in a Javascript environment.
In fact, the authors are not aware of a single Javascript
implementation that can read either format.

Even if you had Javascript implementations for these formats, this
approach would necessitate having to download the entire file into
some kind of ``in memory'' representation to be parsed.  This is
because these formats do not provide a simple way to extract the
required data via a few simple reads.

The other approach, the one we've taken for the \recon\ project, is to
use a format (of our own design, the \meld\ format) that is designed
for remote access.  Given access to header information (more on this
in a moment), we can extract a single table column or single object
with a single read.  Furthermore, we can use the HTTP \code{Range}
header to specify exactly which range of bytes we require.  In this
way, we only have to endure the latency of a single network request
and we avoid transmitting any extraneous data which minimizes the
impact of bandwidth limitations.

% # of rea<ds

Note the previous paragraph presumes we already have the header
information from the files.  The worst case scenario for getting
header information is to make two additional (one time only) network
requests.  The first request would be for the first 18 bytes (again,
utilizing the \code{Range} header) to establish the size of the
header.  The next request would be to read the binary header data
which includes information about the locations of all table columns
and objects.  Once these two requests have been made, the client would
be able to access any object or column with only one additional
request.

% Caching

One of the things we've tried to do in this project is avoid solving
problems that have already been solved.  This was the motivation
behind the use of \msgpack, but we are also assuming that most network
requests for data will be made using HTTP.  This is a safe assumption
given the ubiquitous nature of HTTP (or HTTPS, for that matter).  But
if we assume that requests are made via HTTP we also gain the
additional benefits of caching.

Most networks are already equipped to efficiently handle caching
transparently.  Of course, within a given application, we might
maintain a cache of headers for different results files that we may be
interested in.  But what about multiple users accessing these results
across the same network?  As it turns out, once one user accesses the
file, it is quite likely that the header information will be stored in
a caching proxy between the users network and the storage provider.
This is a caching scenario that we cannot address within an
application, but nevertheless, it is very likely that in such a
multi-user environment the existing network proxies would
transparently act to improve overall performance and enhance the user
experience.

\subsection{Metadata}

% Metadata connected to everything

As discussed throughout Section \ref{sec:spec}, the \wall\ and
\meld\ formats have extensive support for metadata.  This provides a
clean mechanisms for including metadata in files without the need to
mix it in or conflate it with actual data.  Furthermore, metadata is
supported for a wide range of entities represented in the file.

There are many potential applications for such metadata.  For example,
file level metadata can be used to store useful information such as
who ran a simulation, what particular model they ran, what
modifications (if any) were applied to the model or even a complete
listing of all the parameter values that were used to simulate the
model.  In addition, general experimental data could also be stored in
the results file.

Metadata at the signal level could be used to specify not just
descriptions of those variables along with their physical units but,
could also be used to include other attributes like start values,
nominal values, \textit{etc.},

In summary, the \recon\ formats make metadata a first class citizen
within the file formats and this opens the door to many very useful
applications involving metadata.


\subsection{Performance}

It is worth taking a moment to discuss the actual performance of this
approach vs. existing approaches.  As a baseline, we have generated a
``representative results file'' by simulating the R3 robot example
from the Modelica Standard Library.  This results file, produced by
Dymola and written in the ``dsres'' format, will serve as our
baseline.

% Space efficiency

The first issue we will examine is space efficiency.  The baseline
results take up 3,069,623 bytes of storage (for the storage options we
selected).  When stored in the \meld\ format \textbf{without compression}
those same results take up 3,169,994 bytes.  Note that this
translation process from the dsres format to the \meld\ format
preserves variable names and description strings as well as numerical
trajectories.  This means the \meld\ format is only 3.3\% larger than
the corresponding dsres file.  If we enable compression, the
\meld\ file is only 2,787,467 bytes which means it is almost 10\%
smaller than the dsres file.

So, in terms of storage, the \meld\ format is quite comparable to the
dsres format.  At first, this may seem counter-intuitive since we
know, from our earlier discussion of \msgpack, that arrays of floating
point numbers incur an extra byte for storage.  However, the dsres
format cannot store strings very efficiently and that long
descriptions will result in enormous amounts of padding.  So it would
appear that, on net, these two effects cancel each other out.
However, one advantage that the \meld\ format has that the dsres
format doesn't (and an advantage that is not capitalized on in these
benchmarks at all) is the wider range of alias transformations.  In
particular, it is possible to relate two signals via an affine
transformation with the \meld\ format.  Unfortunately, we do yet have
any data on the potential savings this might offer.

% extracting a single signal

Another benchmark worth considering is the time it takes to extract
one particular trajectory from a results file.  For extracting results
from a dsres file, we used the \code{scipy.io.loadmat} function to
load the \code{data\_2} matrix and then extracted the $0^{th}$ column
(Time).  In other words,

{\footnotesize
\begin{verbatim}
mat = loadmat("tests/fullRobot.mat")
T2 = mat["data_2"]
time = T2[0,:]
\end{verbatim}
}

Running this script 5 times gave the following times (in
milliseconds): 18.111, 18.143, 20.076, 19.309, 21.773 for an average time
of 19.482 milliseconds.

In the case of the \meld\ file, we opened the results file, created a
\code{MeldReader} object to read it, extracted the \code{data\_2}
table (called \code{T2} in the translated file) and then extracted the
data associated with the \code{Time} signal.  That code looks as
follows,

{\footnotesize
\begin{verbatim}
with open("dsres_robot.mld", "rb") as fp:
  meld = MeldReader(fp)
  dt = meld.read_table("T2")
  time = dt.data("Time")
\end{verbatim}
}

Running this script 5 times gave the following times (again, in
milliseconds): 17.260, 18.074, 16.882, 14.608, 14.572 for an average
time of 16.280 milliseconds.

It is worth noting the SciPy implementation is very mature and
utilizes \code{numpy} internally.  So one would expect excellent
performance from that implementation.  Nevertheless, extracting data
based in the \meld\ format was over 15\% faster on average.

% reading everything into memory

The last benchmark will be reading multiple signals.  In fact, our
benchmark in this case will be to extract all transient signals into a
Python dictionary.  We will do this again for both formats.  We used
the following script to extract the same signals from the dsres file
format:

{\footnotesize
\begin{verbatim}
ret = {}
mf = dymat.DyMatFile("tests/fullRobot.mat")
for signal in mf.names(2):
    ret[signal] = mf.data(signal)
ret["Time"] = mf.abscissa(2)
\end{verbatim}
}

This code uses the \code{dymat} library which, in turn, leverages
SciPy and numpy.  Running this code 5 times gave the following
execution times (in milliseconds): 519.50, 495.38, 483.60, 476.69 and
481.15 for an average execution time of 491.26 milliseconds.

The script for performing this benchmark using the \meld\ format looks
as follows:

{\footnotesize
\begin{verbatim}
ret = {}
with open("dsres_robot.mld", "rb") as fp:
  meld = MeldReader(fp)
  dt = meld.read_table("T2")
  for signal in dt.signals():
      ret[signal] = dt.data(signal)
\end{verbatim}
}

Running this script 5 times, we record execution times (in
milliseconds) of: 246.34, 240.66, 225.58, 224.22, 222.65 for an
average of 231.89 milliseconds (over 50\% faster compared to the dsres
version of the benchmark).

In fairness, it is worth pointing out that we generally expect
simulation tools to write output results in the \wall\ format first
and then convert them into the \meld\ format.  As such, they will
incur some penalty as a one-time cost when regenerating their results
in the \meld\ format and this penalty is not taken into account here.
Furthermore, this benchmark doesn't include any writing performance
benchmarks (primarily because we have not connected any of these codes
to actual simulation tools to measure write performance).

In summary, we do not have any benchmarks that compare write
performance.  But in terms of storage efficiency, the \meld\ file
format was only 3\% larger when compared to our baseline results in
dsres format.  On the other hand, in terms of read performance, the
\meld\ format was between 15 and 50\% faster depending on the amount
of data to be read.

\subsection{Implementations}

As already mentioned, there is a reference implementation of both the
\wall\ and \meld\ formats already available in Python\cite{pyRecon}.
There are also implementations available for both C\cite{crecon} and
Java\cite{jrecon}.

In addition, the \wall\ format is also now supported by OpenModelica
as of \code{r18784}.  The implementation of the \wall\ format required
408 lines of code to implement in OpenModelica (this includes
implementation of all necessary \msgpack\ operations, \textit{i.e.,}
no external libraries are used to implement \msgpack\ functionality)
while the implementation of the ``dsres'' format uses 656 lines of
code.

\section{Conclusion}
\label{sec:conclusion}

In conclusion, the \recon\ file formats support many important features:
\begin{enumerate}
% easy to implement on a wide range of platforms
\item \textbf{easy to implement on a wide range of platforms} - An open
  source reference implementation is available in Python with
  implementations in C and Java already planned.
% efficient disk and network access
\item \textbf{efficient disk and network access} - These formats have been
  optimized for efficient network performance.  But as the benchmarks
  show, these formats also perform very well for when results are
  stored locally on hard disks.
% first-class metadata
\item \textbf{first-class metadata} - With metadata available at all
  structural levels, results files can embed metadata in a clean and
  practical way.  This has the potential to open up many new and
  interesting applications.
% mixed data-type tables
\item \textbf{tables containing mixed data types} - All data types are
  given equal treatment within the \recon\ formats.  As a result, it
  is possible to embed a wide variety of data and still maintain all
  the other benefits associated with these formats.
% richer alias transformations
\item \textbf{richer alias transformations} - The \recon\ formats support
  not only the common ``inverse'' transform, but a richer set of
  affine transformations.  This could lead to further space
  efficiencies.
% first class and efficient treatment for strings
\item \textbf{efficient treatment of strings} - MATLAB version 4 files are
  very inefficient for storing collections of strings.  By using
  \msgpack\ for serialization, we get very efficient handling of
  string data.
% ability to store objects/structures as well as tables.
\item \textbf{ability to store objects/structures} - Not all data is
  tabular.  For example, parameter data used as input to simulations
  is more naturally represented as an object.  There are many other
  examples of potential applications that need to represent data as
  objects.  The \recon\ formats even allow these objects to be
  embedded within tables.
\end{enumerate}

The \recon\ formats provide all these benefits without any significant
compromise in storage efficiency.

\printbibliography
\end{document}
