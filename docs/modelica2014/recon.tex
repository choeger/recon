% sample file for Modelica Conference paper

\documentclass[11pt,a4paper,twocolumn]{article}
\usepackage{graphicx}
\graphicspath{{fig/}}
\usepackage[T1]{fontenc}
\usepackage[utf8]{inputenc}    %% european characters can be used
\usepackage{lmodern,amsmath,mathptmx,url}      %% recommended for readable pdf
\pagestyle{empty}                %% no page numbers!
\usepackage{geometry}            %% please don't change geometry settings!
\geometry{left=20mm, right=20mm, top=25.4mm, bottom=25mm, noheadfoot, columnsep=8mm}
\parindent0pt
\bibliographystyle{ieeetr}

% some additional packages
\usepackage{listings} % for code listings
\usepackage{color}
\usepackage[hidelinks=true]{hyperref}

% usefull commands
\newcommand{\myr}{\textsuperscript{\textregistered}}
\newcommand{\ud}{\mathrm{d}}
\newcommand{\matx}[1]{\mathbf{#1}}
\newcommand{\recon}{\texttt{recon}}
\newcommand{\wall}{\texttt{wall}}
\newcommand{\meld}{\texttt{meld}}
\newcommand{\msgpack}{\texttt{msgpack}}
\newcommand{\code}[1]{\texttt{#1}} % make quoting code text a bit simpler

\begin{document}

\title{\textbf{{\small Modelica'2014}\\
    \recon -- Web and network friendly simulation data formats}}

\author{Michael Tiller\\
  \href{http://xogeny.com}{Xogeny Inc.}, USA\\
  \href{mailto:michael.tiller@xogeny.com}
       {\nolinkurl{michael.tiller@xogeny.com}}
  \and Peter Harman\\
  \href{http://www.cydesign.com}
       {CyDesign}, UK\\
  \href{mailto:peter@cydesign.com}{\nolinkurl{peter@cydesign.com}}}
\date{} % <--- leave date empty
\maketitle\thispagestyle{empty} %% <-- you need this for the first page

\section*{Abstract}

As an organization, the Modelica Association champions standards in
the area of modeling and simulation.  This effort started with the
Modelica language \cite{ModSpec} and Modelica Standard
Library\cite{MSL} and has recently expanded to include the Functional
Mockup Interface (FMI) standard \cite{FMI} as well.

However, one thing that is conspicuously absent in the Modelica
community is an accepted standard for exchange simulation results
and/or tabular data.  In fact, many tools still rely on a proprietary
format that dates back to MATLAB version 4.  It is remarkably ironic
that most Modelica users are still using this proprietary format while
The MathWorks themselves have abandoned it in favor of an open
standard, HDF5\cite{HDF5}.

This paper will introduce two different file formats.  The first
format we call the \wall format and the other we call the \meld
format.  Both of these were developed as part of a project ultimately
dubbed the \recon project.

It is important to note that we are {\b not} proposing to make these
formats official Modelica standards.  These formats were created from
some specific use cases shared by their developers but not necessary
shared by the majority of Modelica users.  This paper will discuss the
specific use cases which led to the development of these standards and
some of the interesting features of these formats.

In the open tradition of the Mdoelica Association, the authors have
made implementations for these formats available as open source
libraries with the hope that they will benefit the community as a
whole.

\paragraph{Keywords:}\emph{modelica, FMI, simulation results,
  cloud, web, open source}

\section{Introduction}
\label{sec:intro}

Several groups have examined the issue of standardized file formats
\cite{AndreasHDF5,GallHDF5} in the context of Modelica.  In keeping
with the principals of the Modelica Association, an ideal choice would
be a production ready format that is open source and cross-platform.
Most people see HDF5 as a natural choice.  There are already open
source implementations and the file format has been in use for some
time.  As previously mentioned, it has even been adopted by The
MathWorks for use in MATLAB.

But HDF5 has some important drawbacks.  The first is that it is not
truly cross-platform.  The reference implementation of HDF5 is written
in C.  While there are various libraries available for reading HDF5 on
the Java platform\cite{HDFJava}, they are incomplete and awkward to
use.  Another issue with HDF5 is that for simple time-series data it
is over engineered.  HDF5 is feature rich, of that there is no
question, but these features come at the cost of complexity.  This is
why you see very few implementations outside of the reference
implementation.

HDF5 (and many other potential contenders) were developed in an era
when ``compatibility'' for a file format meant you could read and
write it from both FORTRAN and C.  Of course, many simulation tools
are still written using a mixture of FORTRAN and C.  But increasingly,
modeling and simulation are moving more toward web and cloud based
platforms.

In fact, there are many, many standards for manipulating data in web
or cloud based environments.  The most popular format, by far, is
Javascript Object Notation (or JSON, for short).  In addition to other
formats, like XML, you also have a number of different approaches to
marshalling and unmarshalling data for consumption by cross-platform
tools.  Approaches like Google Protocol Buffers\cite{GPB},
Avro\cite{Avro}, Thrift\cite{Thrift} and others are working hard to
try and knit together the various ``big data'' tools like
Hadoop\cite{Hadoop}, Storm\cite{Storm} and ???.

So where does that leave us?  Should we adopt the tried and true
standards from the engineering world and simply live with their lack
of interoperability with important platforms like Java or Javascript?
Or should we try and adapt tools from the ``big data'' world, that
were developed for quite different use cases, to work in the
engineering world.

In some sense, we've chosen a compromise.  As we will see shortly, the
\wall and \meld formats are fundamentally derived from the
\msgpack\cite{msgpack} specification.  This gives us excellent cross
platform compatibility.  But \msgpack is simply a serialization
protocol.  To address some of our more important concerns, to be
discussed shortly, we needed to design a structural context in which
to employ \msgpack. So in this sense, we've created a set of original
file formats that leverage open standards but repurpose them for
modeling and simulation applications.

\section{Goals}
\label{sec:background}

As noted in the 

Problems we want to solve:
\begin{itemize}
\item one
\item another one
\item and a third one
\end{itemize}

\subsection{Exisiting solutions}
\label{sec:exist-sol}

\subsection{New approach}
\label{sec:exist-sol}
Foo
\lstset{language=python}
\begin{lstlisting}[frame=single]  % Start your code-block
f()
\end{lstlisting}


Search for librararies is done by executed by doing:
\lstset{language=bash}
\begin{lstlisting}[frame=shadowbox]  % Start your code-block

% command
\end{lstlisting}

\section{Discussion}
\label{sec:discussion}
%TODO

\section{Conclusion}
\label{sec:conclusion}
%TODO

\bibliography{recon}
\end{document}
