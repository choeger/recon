% sample file for Modelica Conference paper

\documentclass[11pt,a4paper,twocolumn]{article}
\usepackage{graphicx}
\graphicspath{{fig/}}
\usepackage[T1]{fontenc}
\usepackage[utf8]{inputenc}    %% european characters can be used
\usepackage{lmodern,amsmath,mathptmx,url}      %% recommended for readable pdf
\pagestyle{empty}                %% no page numbers!
\usepackage{geometry}            %% please don't change geometry settings!
\geometry{left=20mm, right=20mm, top=25.4mm, bottom=25mm, noheadfoot, columnsep=8mm}
\parindent0pt
\bibliographystyle{ieeetr}

% some additional packages
\usepackage{listings} % for code listings
\usepackage{color}
\usepackage[hidelinks=true]{hyperref}

% usefull commands
\newcommand{\myr}{\textsuperscript{\textregistered}}
\newcommand{\ud}{\mathrm{d}}
\newcommand{\matx}[1]{\mathbf{#1}}
\newcommand{\recon}{\texttt{recon}}
\newcommand{\wall}{\texttt{wall}}
\newcommand{\meld}{\texttt{meld}}
\newcommand{\code}[1]{\texttt{#1}} % make quoting code text a bit simpler

\begin{document}

\title{\textbf{{\small Modelica'2014}\\
    \recon -- Web and network friendly simulation data formats}}

\author{Michael Tiller\\
  \href{http://xogeny.com}{Xogeny Inc.}, USA\\
  \href{mailto:michael.tiller@xogeny.com}
       {\nolinkurl{michael.tiller@xogeny.com}}
  \and Peter Harman\\
  \href{http://www.cydesign.com}
       {CyDesign}, UK\\
  \href{mailto:peter@cydesign.com}{\nolinkurl{peter@cydesign.com}}}
\date{} % <--- leave date empty
\maketitle\thispagestyle{empty} %% <-- you need this for the first page

\section*{Abstract}


\paragraph{Keywords:}\emph{modelica, FMI, simulation results,
  cloud, web, open source}

\section{Introduction}
\label{sec:intro}

\section{Background}
\label{sec:background}

Problems we want to solve:
\begin{itemize}
\item one
\item another one
\item and a third one
\end{itemize}

\subsection{Exisiting solutions}
\label{sec:exist-sol}

\subsection{New approach}
\label{sec:exist-sol}
Foo
\lstset{language=python}
\begin{lstlisting}[frame=single]  % Start your code-block
f()
\end{lstlisting}


Search for librararies is done by executed by doing:
\lstset{language=bash}
\begin{lstlisting}[frame=shadowbox]  % Start your code-block

% command
\end{lstlisting}

\section{Discussion}
\label{sec:discussion}
%TODO

\section{Conclusion}
\label{sec:conclusion}
%TODO

\bibliography{recon}
\end{document}
