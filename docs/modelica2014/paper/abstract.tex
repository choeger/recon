%sample file for Modelica 2011 Abstract page

\documentclass[11pt,a4paper]{article}
\usepackage{graphicx}
\graphicspath{{fig/}}
% uncomment according to your operating system:
% ------------------------------------------------
%\usepackage[latin1]{inputenc}    %% european characters can be used (Windows, old Linux)
\usepackage[utf8]{inputenc}     %% european characters can be used (Linux)
%\usepackage[applemac]{inputenc} %% european characters can be used (Mac OS)
% ------------------------------------------------
\usepackage[T1]{fontenc}   %% get hyphenation and accented letters right
\usepackage{mathptmx}      %% use fitting times fonts also in formulas
\usepackage{lmodern,amsmath,mathptmx,url}      %% recommended for readable pdf
% do not change these lines:
\pagestyle{empty}                %% no page numbers!
\usepackage[left=35mm, right=35mm, top=15mm, bottom=20mm, noheadfoot]{geometry}
%% please don't change geometry settings!
\setlength{\parindent}{10pt}
\setlength{\parskip}{3 mm}

% some additional packages
\usepackage{color}
\usepackage{indentfirst}
\usepackage[hidelinks=true]{hyperref}
\usepackage[hidelinks=true]{hyperref}
\hypersetup{%
  pdftitle = {recon  -- Web and network friendly simulation data formats},
  pdfauthor = {Michael Tiller and Peter Harman},
  pdfsubject = {10th International Modelica Conference 2014},
  pdfkeywords = {Modelica, FMI, simulation results, cloud, web, open source}}
\usepackage[backend=bibtex,sorting=none]{biblatex}
\addbibresource{recon}

% usefull commands
\newcommand{\recon}{\texttt{recon}}
\newcommand{\wall}{\texttt{wall}}
\newcommand{\meld}{\texttt{meld}}
\newcommand{\msgpack}{\texttt{msgpack}}
\newcommand{\code}[1]{\texttt{#1}} % make quoting code text a bit simpler

\newcommand{\myr}{\textsuperscript{\textregistered}}
\newcommand{\ud}{\mathrm{d}}
\newcommand{\matx}[1]{\mathbf{#1}}

% begin the document
\begin{document}
\thispagestyle{empty}

\title{\recon\  -- Web and network friendly simulation data formats}

\author{Michael Tiller\\
  \href{http://xogeny.com}{Xogeny Inc.}, USA\\
  \href{mailto:michael.tiller@xogeny.com}
       {\nolinkurl{michael.tiller@xogeny.com}}
  \and Peter Harman\\
  \href{http://www.cydesign.com}
       {CyDesign Ltd.}, UK\\
  \href{mailto:peter@cydesign.com}{\nolinkurl{peter@cydesign.com}}}
\date{} % <--- leave date empty
\maketitle\thispagestyle{empty} %% <-- you need this for the first page

There are many different commonly used file formats for storing time
series data.  Most of these file formats are designed with the
assumption that the file itself will be locally available to the
software that will be reading or writing the data stored in them.
While this assumption is an excellent one for desktop based tools with
direct access to disk drives capable of moving virtually
instantaneously around from sector to sector, there are a growing
number of applications for which local access is not necessarily
available.  For these applications, we've initiated the
\recon\ project to develop more suitable formats.

With the emergence of web and cloud based modeling and simulation
technologies, the time has come to explore file formats specifically
optimized for non-desktop applications.  In this paper, we present a
new set of file formats that are specifically designed for web and
cloud based approaches.  This paper reviews the key requirements for
web and cloud enabled applications and then presents a specification
for two file formats that address those requirements.

When considering the various use cases that drove our requirements, we
recognized that two different file formats were actually required.
The first format, the \wall\ format, is optimized for writing.  The
other format, the \meld\ format, is optimized for reading over a
network (\textit{i.e.,} minimizing the number of reads and bytes
read).  We will discuss the layout of each of these formats and
describe the use cases for which they are most appropriate.

In the open tradition of the Modelica Association, the authors have
made specifications and implementations for these formats available as
open source libraries with the hope that they will benefit the
community as a whole.

\printbibliography

\end{document}
